\documentclass{polytech/polytech}
\usepackage{amsmath}
% zone du préambule

\typereport{custom}
\typereportname{Rapport de projet développement DII3}

\reportyear{2017-2018}
\schooldepartment{dii}

\title{Outil d'aide à l'affectation des projets}

\student[dii3]{Jérémy}{Loche}{jeremy.loche@etu.univ-tours.fr}
\student[dii3]{Louis}{Thomas}{louis.thomas@etu.univ-tours.fr}

\academicsupervisor[dii]{Ameur}{Soukhal}{ameur.soukhal@univ-tours.fr}

\motcle{affectation}
\motcle{solver}
\motcle{modèle}
\motcle{mathématique}
\motcle{google}
\motcle{sheet}
\resume{Ce projet a pour but de fournir un outil d'aide a l'affectation des projets.}

\keyword{example}
\keyword{\LaTeX}
\abstract{This is an example of an abstract for the report.}

\posterblock{Constat}{Plein de truc sont à améliorer dans ce projet}{images/modeemploi}{}
\posterblock{Solution}{Il y en a plein :
\begin{itemize}
    \item la première
    \item la deuxième
\end{itemize}
}{images/fig1}{}

\posterblock{Conclusion}{Tout marche nickel}{images/fig2}{Ma légende}

\addbibresource{biblio}

\begin{document}
% zone du contenu du document

%=========BEGIN chap 0=============
\chapter{Introduction}
\label{chap:intro}

%=========BEGIN chap 1=============
\chapter{Le problème d'affectation}
\label{chap:pb_affectation}

\section{Présentation}
\label{sec:pres_affectation}

Ce travail est un problème d'affectation.
En effet, le but est d'associer à un ou plusieurs individus un projet précis en respectant un objectif simple: maximiser la satisfaction générale de l'affectation.
L'idée et que chacun soit satisfait du projet auquel il a été affecté et qu'il soit en mesure de le mener à bien dans les meilleures conditions.


\section{La modélisation mathématique}
\label{sec:mod_math}

Pour résoudre ce problème, nous allons avoir besoin de la formaliser de manière mathématique afin d'essayer de trouver des solutions.
Nous rappelons que le but est d'affecter un étudiant, un binôme ou un groupe de personnes à un projet.

\subsection{Les données et les paramètres}
\label{sec:donnees_params}
Pour cela, il faut créer deux entités appartenant à deux ensembles qui seront nos données de départ:
\begin{enumerate}
\item une entité $individu \in Individus$ : représente une personne, un binôme ou un groupe de personnes qui accomplira un \textbf{projet};
\item une entité  $projet \in Projets$ : représente un projet auquel sera affecté un \textbf{individu}.
\end{enumerate}

Pour plus de simplicité, on va indicer les entités comme dans le tableau \ref{tab:indice_entite_mod_math}. On a choisi que les binômes seront représentés par des individus.
\begin{table}
\caption{Indiçage des entités du modèle mathématique}
\label{tab:indice_entite_mod_math}
\begin{tabular}{|c|c|c|c|}
\hline 
\multicolumn{2}{|c|}{$\forall individu \in Individus$} & \multicolumn{2}{|c|}{$\forall projet \in Projets$} \\ 
\hline 
\textbf{Indice Individu} & Nom individu &\textbf{ Indice Projet} & Nom projet \\ 
\hline 
\textbf{1} & Binôme: Arthur, Léo & \textbf{1} & Projet: Lampe connecté \\ 
\hline 
\textbf{2} & Binôme: Sophie, Jean & \textbf{2} & Projet: Moniteur UVA \\ 
\hline 
... & ... & ... & ... \\ 
\hline 
\textbf{nbIndividus} & Binôme: Paul, Pierre &\textbf{ nbProjets} & Projet: Voiture RC \\ 
\hline 
\end{tabular} 
\end{table}

Pour que l'affectation de chaque individu à un projet, il faut assez de projet pour tout le monde ce qui implique la relation suivante:
$$ nbProjets \geqslant nbIndividus$$

Pour tout les individus, on souhaite connaître sa préférence pour un projet afin de procéder à l'affectation. En utilisant le formalisme mathématique, on peut définir la quantité $preference$ qui définira l'appréciation d'un individu à un projet donné:
$$
\forall (i,p) \in Individus \times Projets, 
preference[i,p] \in [1, nbProjets]
$$
On choisit de dire que cette préférence va de $nbProjet$ à 1 dans l'ordre décroissant d'appréciation. Ainsi, un individu qui aurait la préférence $nbProjets$ pour un projet indiquera que ce projet est son préféré et 1 pour celui qui lui plait le moins.

Pour que l'affectation soit équitable, un individu ne peut pas donner la même préférence à deux projets différents. De manière mathématique cela donne:
$$
\forall (i,p1,p2) \in Individus \times Projets^2 , p1 \neq p2 \Rightarrow preference[i,p1] \neq preference[i,p2]
$$

Lorsque ces pré-conditions sont respectés, alors on crée un \textbf{paramètre} appelé préférence utile pour résoudre le problème d'affectation.
$$
\forall (i,p) \in Individus \times Projets, 
preference[i,p]
$$

Nous allons maintenant voir quels contraintes et variables sont à déterminer pour résoudre le problème.

\subsection{Les variables et les contraintes}

Pour modéliser entièrement le problème d'affectation, la préférence de chaque individus pour un projet n'est pas suffisante. 
Nous allons devoir mettre en place des \textbf{ variables} et des \textbf{contraintes} qu'elles doivent respecter.

Une \textbf{variable} est une valeur qui doit être déterminée pour résoudre le problème.

Nous cherchons à déterminer a quel projet est affecté un individu. Pour cela on crée la variable $affectation$:
$$
\forall (i,p) \in Individu\times Projets , affectation[i,p]=
\left\lbrace
\begin{array}{l}
1 \text{ si l'individu i est affecté au projet p} \\
0 \text{ sinon}
\end{array} 
\right.
$$
	
\subsection{Solution simple, maximiser la préférence moyenne}
\label{sec:max_pref_moy}


\subsection{Solution de compromis, la préférence minimale}
\label{sec:pref_min}
%=========END chap 1================

\end{document}
