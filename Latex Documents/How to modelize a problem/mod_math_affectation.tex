\documentclass{article}

\usepackage{amsmath}
\usepackage[french]{babel}
\usepackage[utf8]{inputenc}
\usepackage[T1]{fontenc}
\usepackage{stmaryrd}

\title{Modèle mathématique pour l'affectation de projets}
\date{19 octobre 2017}
\author{
LOCHE Jérémy\\ Département Informatique industrielle 3A, Polytech Tours
\and
THOMAS Louis\\ Département Informatique industrielle 3A, Polytech Tours}



\begin{document}
\maketitle

\newpage

\section{Le problème en français}
On dispose d'une liste d'étudiants et d'une liste de projets. On suppose qu'il y a au moins autant de projets que d'étudiants ou de groupes minimum d'étudiants attribuable à un projet. Chaque étudiant confectionne une liste de projets ordonnée par ordre décroissant de préférence sur lequel il souhaite travailler.\' A chaque projet ne peuvent être affecté qu'un nombre précis d'étudiants. Un étudiant ne peut être affecté qu'à un et un seul projet.


\section{L'interprétation mathématique}

Il faut traduire tout ça dans un modèle mathématique. Nous choisissons de représenter les étudiants par un graphe bipartie dans lequel $n$ noeuds "étudiants"  et $m$ noeuds "projets"   sont relié par des arcs pondéré par un nombre entier allant de 1 à "nombre~de~projets".

On noter par la suite:
$$
i \in \llbracket 1,n \rrbracket 
\\
\text{ }
i \Rightarrow \text{etudiant}
$$

$$
j \in \llbracket 1,m \rrbracket 
\\
\text{ }
j \Rightarrow \text{projet}
$$

$$
A=
\left(
\begin{array}{c c c c}
a_{11} & ...        & a_{1m} \\ 
...        & a_{ij}   & ... \\
a_{n1} & ...       & a_{nm} \\
\end{array}
\right)
\Rightarrow \text{Matrice des voeux}
$$

$$
X=
\left(
\begin{array}{c c c c}
x_{11} & ...        & x_{1m} \\ 
...        & x_{ij}   & ... \\
x_{n1} & ...       & x_{nm} \\
\end{array}
\right)
\Rightarrow \text{Matrice d'affectation aux projets}
$$

$$
a_{ij} \in  \llbracket 1,m \rrbracket \Rightarrow \text{voeux de l'étudiant i pour le projet j avec m le choix préféré et 1 le pire}
$$

$$
x_{ij} \in  \lbrace 1,m \rbrace
\Rightarrow
x_{ij}=
\left\lbrace
\begin{array}{l}
1 \text{ : le projet j est affecté à l'étudiant i} \\
0 \text{ : sinon}
\end{array}
\right.
$$


\end{document}